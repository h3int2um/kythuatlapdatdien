 \documentclass[20pt]{beamer}
\usepackage[utf8]{inputenc}
\usepackage[utf8]{vietnam}
\usepackage{amsmath}
\usepackage{amsfonts}
\usepackage{amssymb}
\usepackage{graphicx}
\usepackage{xcolor}
\usepackage{utopia} %font utopia imported

\usepackage{ragged2e}
\usepackage{etoolbox}

\mode<beamer>{\usetheme{CambridgeUS}}

\usecolortheme{default}

\usepackage{hyperref}
\hypersetup{pdfpagemode=FullScreen} %mode FullScreen with beamer

\apptocmd{\frame}{}{\justifying}{} % Allow optional arguments after frame.

\usepackage{comment}

\makeatletter
\let\insertuniversity\relax
\newcommand\universitytitle{TRƯỜNG ĐH}

\let\insertclass\relax
\newcommand\classtitle{Lớp}

\let\insertcourse\relax
\newcommand\coursetitle{Môn học}

\mode<all>
{
  \newcommand\university[1]{\def\insertuniversity{#1}}
  
  \newcommand\class[1]{\def\insertclass{#1}}
  
  \newcommand\course[1]{\def\insertcourse{#1}}
  \titlegraphic{}
}

\defbeamertemplate*{title page}{supdefault}[1][]
{
  \begingroup
    \centering
    \ifx\insertuniversity\relax\relax\else
    \begin{beamercolorbox}[sep=2pt,center,#1]{author}
      \footnotesize\universitytitle~\insertuniversity
    \end{beamercolorbox}\fi
    
    \vspace{-.15cm}
    \begin{beamercolorbox}[sep=8pt,center,#1]{title}
      \usebeamerfont{title}\Large\inserttitle\par%
      \ifx\insertsubtitle\@empty\relax%
      \else%
        \vskip0.25em%
        {\usebeamerfont{subtitle}\usebeamercolor[fg]{subtitle}\insertsubtitle\par}%
      \fi%     
    \end{beamercolorbox}%
    \vskip.5em\par

    \vspace{-.35cm}
    \ifx\insertcourse\relax\relax\else
    \begin{beamercolorbox}[sep=6pt,center,#1]{author}
      \usebeamerfont{author}\small\coursetitle:~\insertcourse
    \end{beamercolorbox}\fi
    \vspace{-.35cm}
    \ifx\insertclass\relax\relax\else
    \begin{beamercolorbox}[sep=6pt,center,#1]{author}
      \usebeamerfont{author}\small\classtitle:~\insertclass
    \end{beamercolorbox}\fi

    \vspace{-.35cm}
    \begin{beamercolorbox}[sep=6pt,center,#1]{author}
      \usebeamerfont{author}\small\insertauthor
    \end{beamercolorbox}
    %\begin{beamercolorbox}[sep=8pt,center,#1]{institute}
      %\usebeamerfont{institute}\insertinstitute
    %\end{beamercolorbox}
    \vspace{-.45cm}
    \begin{beamercolorbox}[sep=8pt,center,#1]{date}
      \usebeamerfont{date}\small\insertdate
    \end{beamercolorbox}\vskip0.5em
    {\usebeamercolor[fg]{titlegraphic}\inserttitlegraphic\par}
  \endgroup
  \vfill
}
\setbeamertemplate{title page}[supdefault][colsep=-4bp,rounded=true,shadow=\beamer@themerounded@shadow]\makeatother

%Title page
%\title[Trang bị phân phối và TBA]{\emph{Chủ đề báo cáo}\\\textbf{Trang bị phân phối và Trạm biến áp}}
\title[Trang bị phân phối và TBA]{\emph{Chủ đề báo cáo}\\\textbf{TRANG BỊ PHÂN PHỐI VÀ TRẠM BIẾN ÁP
}}
\author[Kỹ thuật lắp đặt điện]{GVHD: Huỳnh Phát Triển \and SVTH: Nhóm 4}
\course{Kỹ thuật lắp đặt điện}
\class{Công nghệ, kỹ thuật điện, điện tử}
\university{KỸ THUẬT -- CÔNG NGHỆ CẦN THƠ}
\date[Nhóm 4]{\today}
%\date[Nhóm 1]{Ngày 24 tháng 08 năm 2016}

%\logo{\includegraphics[height=1.3cm]{logo_ctut.pdf}}

\AtBeginSection[]
{
  \begin{frame}
    \frametitle{Nội dung báo cáo}
    \justifying
    \tableofcontents[currentsection]
  \end{frame}
}
\definecolor{doden}{RGB}{204, 0, 0}
\newcommand{\noibat}[1]{\textcolor{blue}{#1}} % Làm nổi bật text quan trọng.
\newcommand{\tviet}[2]{#1_{\text{\textit{#2}}}}
\begin{document}
%http://tex.stackexchange.com/questions/82794/removing-page-number-from-title-frame-without-changing-the-theme
\bgroup
\makeatletter
\setbeamertemplate{footline}
{
  \leavevmode%
  \hbox{%
  \begin{beamercolorbox}[wd=.333333\paperwidth,ht=2.25ex,dp=1ex,center]{author in head/foot}%
    \usebeamerfont{author in head/foot}\insertshortauthor\expandafter\beamer@ifempty\expandafter{\beamer@shortinstitute}{}{~~(\insertshortinstitute)}
  \end{beamercolorbox}%
  \begin{beamercolorbox}[wd=.333333\paperwidth,ht=2.25ex,dp=1ex,center]{title in head/foot}%
    \usebeamerfont{title in head/foot}\insertshorttitle
  \end{beamercolorbox}%
  \begin{beamercolorbox}[wd=.333333\paperwidth,ht=2.25ex,dp=1ex,right]{date in head/foot}%
    \usebeamerfont{date in head/foot}\insertshortdate{}\hspace*{2em}
%    \insertframenumber{} / \inserttotalframenumber\hspace*{2ex} 
    \hspace*{6ex}
  \end{beamercolorbox}}%
  \vskip0pt%
}

\begin{frame}
\titlepage
\end{frame}
\egroup

\setcounter{framenumber}{0}

%--------------------------------------------------------------------------------
%--------------------------------------------------------------------------------
% Danh sách thành viên
\begin{frame}{Danh sách thành viên}
	\vspace{-1cm}
	\begin{small}
	\begin{columns}
		\column{0.6\textwidth}
		\begin{enumerate}			
			\item Nguyễn Văn Đình
			\item Nguyễn Hoàn Hận
			\item Thi Minh Nhựt
			\item Phạm Thanh Quý
			\end{enumerate}

		\column{0.6\textwidth}
		\begin{enumerate}	% Danh sách tiếp theo
			\setcounter{enumi}{4}
			\item Liên Thái Trường
			\item Lư Anh Tuấn
			\item Phan Thành Tuấn
			\item[]
		\end{enumerate}
	\end{columns}
	\end{small}
\end{frame}

%--------------------------------------------------------------------------------
%--------------------------------------------------------------------------------
% Nội dung báo cáo
\begin{frame}	% Trang mục lục
	\frametitle{Nội dung báo cáo}
	\tableofcontents
\end{frame}

%--------------------------------------------------------------------------------
%--------------------------------------------------------------------------------
% Phạm vi của chủ đề báo cáo
\section*{Phạm vi của chủ đề báo cáo}

\begin{frame}{Cấp điện áp}
	\begin{block}{Tìm hiểu trong phạm vi của cấp điện áp}
		\begin{itemize}
			\item \alert{Mạng hạ áp:} đến $1kV$.
			
			\item \alert{Mạng trung áp:} $22kV$.
		\end{itemize}
	\end{block}
\end{frame}

%--------------------------------------------------------------------------------
%--------------------------------------------------------------------------------
% Trang bị phân phối
\section{Trang bị PP điện đến cấp điện áp $1kV$}
\begin{frame}{Trang bị PP đến điện áp $1kV$}
	\begin{itemize}
	\justifying
		\item Quy định chung: phạm vi, yêu cầu.
		
		\item Lắp đặt trang bị điện.
		
		\item Quy định về Thanh cái, Dây dẫn và Cáp điện.
		
		\item Kết cấu của trang bị PP điện.
		
		\item Lắp đặt trang bị PP điện: gian điện, gian sản xuất, ngoài trời.
	\end{itemize}
\end{frame}


%--------------------------------------------------------------------------------
% Quy định chung
\subsection*{Quy định chung}
\begin{frame}{Phạm vi áp dụng}
	\begin{itemize}
	\justifying
		\item Điện áp AC: đến $1kV$.
		
		\item Điện áp DC: đến $1.5kV$.
		
		\item Tủ PP, điều khiển, relay và các đầu ra từ thanh cái: được lắp đặt trong nhà và ngoài trời.		
	\end{itemize}
\end{frame}

\begin{frame}{Yêu cầu chung}
	\begin{itemize}
	\justifying
		\item Lựa chọn thiết bị: theo ĐK làm việc bình thường và khi ngắn mạch.
		
		\item Các mạch, bảng điện trong tủ PP: ghi nhiệm vụ rõ ràng, cụ thể.
		
		\item Bố trí các mạch rõ ràng: mạch AC, mạch DC, cấp điện áp,\ldots có thể phân biệt được.	
		
	\end{itemize}
\end{frame}

\begin{frame}{Yêu cầu chung (tt)}
	\begin{itemize}
	\justifying
		\item Vị trí các pha và các cực: bố trí giống nhau. Với tủ PP: phải có chỗ nối đất di động.
		
		\item Các bộ phận kim loại: sơn, mạ hoặc phủ lớp chống ăn mòn.
		
		\item Nối đất đúng quy định.
		
	\end{itemize}
\end{frame}


%--------------------------------------------------------------------------------
% Lắp đặt trang bị điện
\subsection*{Lắp đặt trang bị điện}
\begin{frame}{Lắp đặt trang bị điện}
	\begin{itemize}
	\justifying
		\item Cách bố trí: an toàn cho người vận hành, và thiết bị xung quanh.
		
		\item Phần động của thiết bị đóng ngắt: không mang điện áp sau khi ngắt điện.
		
		\item Ghi rõ vị trí đóng và vị trí ngắt trên thiết bị đóng ngắt.				
	\end{itemize}
\end{frame}

\begin{frame}{Lắp đặt trang bị điện (tt)}
	\begin{itemize}
	\justifying
		\item Có dự tính khả năng cắt điện cho từng aptomat.
		
		\item Với cầu chì đuôi xoáy: dây nguồn nối vào đuôi, dây tải nối vào vỏ cầu chì.
	\end{itemize}
\end{frame}


%--------------------------------------------------------------------------------
% Thanh cái, dây dẫn, cáp điện
\subsection*{Thanh cái, dây dẫn, cáp điện}
\begin{frame}{Thanh cái, dây dẫn, cáp điện}
\justifying
	Khoảng cách giữa phần \alert{dẫn điện} \noibat{không bọc cách điện} với phần \alert{không dẫn điện}, \noibat{không bọc cách điện}:
	\begin{itemize}
	\justifying
		\item Nhỏ hơn $20mm$ theo bề mặt cách điện và $12mm$ trong không khí.
		\item Rào chắn: nhỏ hơn $100mm$ với rào lưới và $40mm$ với rào tấm kín.
	\end{itemize}
\end{frame}

\begin{frame}{Thanh cái, dây dẫn, cáp điện (tt)}
	\begin{itemize}
	\justifying
		\item \alert{Dây dẫn} và \alert{thanh dẫn trần nối đất} có thể \noibat{không cần cách điện}.
		\item Các mạch điều khiển, đo lường, cách bố trí cáp phải theo yêu cầu quy định.
	\end{itemize}
\end{frame}


%--------------------------------------------------------------------------------
% Kết cấu của trang bị phân phối
\subsection*{Kết cấu của trang bị PP}
\begin{frame}{Kết cấu của trang bị PP}
	\begin{itemize}
	\justifying
		\item \alert{Khung bảng điện} làm bằng \noibat{vật liệu không cháy}, \alert{vỏ và các bộ phận khác} làm bằng vật liệu \noibat{không cháy} hoặc \noibat{khó cháy}.
		
		\item \alert{TBPP được bố trí và lắp đặt} sao cho \noibat{không chịu ảnh hưởng} của tác động bên ngoài.
	\end{itemize}
\end{frame}

\begin{frame}{Kết cấu của trang bị PP (tt)}
	\begin{itemize}
	\justifying
		\item \alert{Bảo vệ các thiết bị chống sự phá hủy của môi trường:} độ ẩm, bụi, đặt ngoài trời,\ldots
	\end{itemize}
\end{frame}


%--------------------------------------------------------------------------------
% Lắp đặt trang bị phân phối trong gian điện
\subsection*{Lắp đặt TBPP trong gian điện}
\begin{frame}{TBPP trong gian điện}
\justifying
	Quy định về hành lang vận hành:
	\begin{itemize}
	\justifying
		\item Rộng $\geq 0.8m$ và cao $\geq 1.9m$.
		\item Những chỗ cá biệt: rộng $\geq 0.6m$	
	\end{itemize}
\end{frame}

\begin{frame}{TBPP trong gian điện (tt)}
\justifying
\alert{Bộ phận mang điện}, \noibat{không bọc cách điện, không rào chắn} đến \alert{bộ phận mang điện}, \noibat{không bọc cách điện, có rào chắn}:
	\begin{itemize}
	\justifying
		\item $U < 660V$: $1m$ (tủ bảng dài $7m$) hoặc $1.2m$ (tủ bảng dài hơn $7m$).
		
		\item $U \geq 660V$: $1.5m$
	\end{itemize}
\end{frame}

\begin{frame}{TBPP trong gian điện (tt)}
\justifying
\noibat{Khoảng cách nhỏ nhất} giữa các \alert{bộ phận mang điện}, \noibat{không bọc cách điện, không rào chắn} đặt ở độ cao $<2.2m$ về 2 phía của lối đi lại:
	\begin{itemize}
	\justifying
		\item $U < 660V$: $1.5m$
		
		\item $U \geq 660V$: $2m$
	\end{itemize}
\end{frame}

\begin{frame}{TBPP trong gian điện (tt)}
	\begin{itemize}
	\justifying
		\item Nếu khoảng cách nhỏ hơn: thì phải đặt rào chắn.
		
		\item Nếu không đặt rào chắn: đặt ở vị trí cao $ \geq 2.2m$
		
		\item Kích thước rào chắn: $\leq 25\times 25 mm^2$, cao $\geq 1.7m$
	\end{itemize}
\end{frame}

\begin{frame}{TBPP trong gian điện (tt)}
	\begin{itemize}
	\justifying
		\item Chiều dài bảng tủ $\geq 7m$: làm 2 lối ra, có thể tự mở từ bên trong.
		\item Chiều dài $\geq 7m$, chiều rộng $>3m$, không dùng dầu: không cần làm cửa thứ 2.
		\item Kích thước cửa: chiều rộng $\geq 0.75m$, chiều cao $\geq 1.9m$
	\end{itemize}
\end{frame}

%--------------------------------------------------------------------------------
% Lắp đặt trang bị phân phối trong gian sản xuất
\subsection*{Lắp đặt TBPP trong gian sản xuất}
\begin{frame}{TBPP trong gian sản xuất}
	\begin{itemize}
	\justifying
		\item Đặt rào chắn: cao $\geq 1.7m$. Chỉ có thể tháo rào bằng dụng cụ chuyên dụng.
		
		\item Khoảng cách từ \alert{rào} đến \alert{bộ phận mang điện}: $\geq 0.7m$
		
		\item Đoạn cuối của dây dẫn nằm gọn trong tủ hoặc thiết bị.
	\end{itemize}
\end{frame}

%--------------------------------------------------------------------------------
% Lắp đặt trang bị phân phối ngoài trời
\subsection*{Lắp đặt TBPP ngoài trời}
\begin{frame}{Lắp đặt TBPP ngoài trời}
	\begin{itemize}
	\justifying
		\item Thiết bị \alert{bố trí trên nền phẳng}, cao $\geq 0.3m$ hoặc $\geq 0.5m$ (tủ bảng).
		\item Có thể \alert{bố trí sấy} đảm bảo hoạt động bình thường cho: thiết bị, relay, khí cụ đo đếm,\ldots
	\end{itemize}
\end{frame}
%--------------------------------------------------------------------------------
%--------------------------------------------------------------------------------
% Trạm biến áp
\section{Trạm biến áp}
\begin{frame}{Trạm biến áp}
	\begin{itemize}
	\justifying
		\item Quy định chung: phạm vi và yêu cầu.
		
		\item TBPP và TBA: ngoài trời, trong nhà.
		
		\item Trạm biến áp: phân xưởng, trên cột.
		\item Bảo vệ chống sét.
		
		\item Hệ thống dầu MBA.
		
		\item Lắp đặt MBA lực.
	\end{itemize}
\end{frame}


%--------------------------------------------------------------------------------
% Phạm vi áp dụng
\subsection*{Phạm vi áp dụng}
\begin{frame}{Phạm vi áp dụng}
	\begin{itemize}
	\justifying
		\item Điện áp AC: $\geq 1kV$.
		
		\item Áp dụng TBA ngoài trời, trong nhà.	
	\end{itemize}
\end{frame}

%--------------------------------------------------------------------------------
% Yêu cầu chung
\subsection*{Yêu cầu chung}
\begin{frame}{Yêu cầu chung}	
	\begin{itemize}
	\justifying
		\item Lựa chọn theo điều kiện làm việc: \alert{bình thường}, \alert{không bình thường}, \alert{khi sửa chữa}.
		
		\item Vận chuyển dễ dàng và an toàn.
		
		\item Chọn TB điện, phần dẫn điện, cách điện: ĐK \alert{ổn định động} và \alert{ổn định nhiệt}, \alert{khả năng đóng cắt} (máy cắt).
	\end{itemize}
\end{frame}


\begin{frame}{Yêu cầu chung (tt)}	
	\begin{itemize}
	\justifying
		\item Dao cách ly: chổ cắt nhìn thấy được nhìn thấy bằng mắt thường.
		
		\item Máy cắt: phải chỉ rõ vị trí đóng ngắt. \alert{Không sử dụng đèn chỉ trạng thái máy cắt.}
	\end{itemize}
\end{frame}

\begin{frame}{Yêu cầu chung (tt)}
\justifying
	Nơi điều khiện xấu, ảnh hưởng TBĐ:
	\begin{itemize}
	\justifying
		\item Tăng cường cách điện.
			
		\item Chọn vật liệu chịu được tác động môi trường hoặc sơn bảo vệ.
			
		\item Tránh hướng gió tác hại, chống bụi, hơi nước.
			
		\item Sơ đồ đơn giản, kiểu kín.
	\end{itemize}
\end{frame}

\begin{frame}{Yêu cầu chung (tt)}
	\begin{itemize}
	\justifying
		\item 	\alert{Thanh dẫn} dùng dây nhôm, nhôm lõi thép, ống hoặc thanh nhôm, hợp kim nhôm, đồng hoặc thanh đồng, hợp kim đồng.
		
		\item Có \alert{khóa cơ khí hoặc điện từ có tính liên động} ở các phần tử loại trừ những sự cố vô tình khi thao tác.
	\end{itemize}
\end{frame}

\begin{frame}{Yêu cầu chung (tt)}
	\begin{itemize}
	\justifying
		\item TBPP và TBA trên $1kV$: dùng dao nối đất cố định, không dùng nối đất di động.
		\item Nên bố trí dao nối đất kết hợp với dao cách ly của máy biến điện áp thanh cái hoặc máy cắt liên lạc .
	\end{itemize}
\end{frame}

\begin{frame}{Yêu cầu chung (tt)}
\justifying
	Chiều cao của rào chắn bảo vệ TBPP và TBA:
	\begin{itemize}
	\justifying
		\item Đặt ngoài trời: $1.8m$
		\item Đặt trong nhà: $1.9m$
		\item Có thể dùng thanh chắn: cao $1.2m$
	\end{itemize}
\end{frame}

\begin{frame}{Yêu cầu chung (tt)}
	\begin{itemize}
	\justifying
		\item Thiết bị đo các thông số của dầu: đặt ở vị trí quan sát thuận lợi, an toàn, không cắt điện khi quan sát,\ldots
		\item Các dây dẫn liên quan đến thiết bị có dầu: chọn \noibat{dây có cách điện chịu dầu}.
		
		\item \alert{Thao tác} bằng \noibat{nguồn AC}.
	\end{itemize}
\end{frame}

\begin{frame}{Yêu cầu chung (tt)}
	\begin{itemize}
	\justifying
		\item \alert{Các TB điện đặt ngoài trời:} \noibat{sơn màu sáng} để giải nhiệt.
		
		\item Hệ thống chiếu sáng: \noibat{phải sử dụng nguồn điện}, đảm bảo vận hành an toàn và thuận tiện.
		
		\item Chú ý về \noibat{khả năng vận chuyển TB}.
	\end{itemize}
\end{frame}

\begin{frame}{Yêu cầu chung (tt)}
	\begin{itemize}
	\justifying
		\item Rào chắn với TBA đặt ngoài trời: cách $1.8m$ và cao trên $1.8m$
		\item \alert{Không sử dụng rào chắn}: \noibat{TBA trong nhà}, \noibat{TBA hợp bộ kiểu kín}, \noibat{TBA trên cột}.
		\item Kết cấu kim loại của trạm: \noibat{chống ăn mòn}.
	\end{itemize}
\end{frame}


%--------------------------------------------------------------------------------
% Trang bị phân phối và TBA ngoài trời
\subsection*{Trang bị phân phối và TBA ngoài trời}
\begin{frame}{Trang bị PP, TBA ngoài trời}
\justifying
	Phần Trang bị phân phối và Trạm biến áp ngoài trong tài liệu quy định cho các cấp điện áp cao, ngoài phạm vi báo cáo.
\end{frame}


%--------------------------------------------------------------------------------
% Trang bị phân phối và TBA phân xưởng
\subsection*{Trang bị phân phối và TBA phân xưởng}
\begin{frame}{TBPP, TBA phân xưởng}
	\begin{itemize}
	\justifying
		\item \alert{Cho phép} đặt trong môi trường \noibat{nhiều bụi}, \noibat{hóa chất độc hại} nhưng cần thực hiện \alert{các biện pháp an toàn}.
		
		\item Khi đặt hở trong gian điện: các phần điện của MBA được che kín, TBPP đặt trong tủ kín và bảo vê.
	\end{itemize}
\end{frame}

\begin{frame}{TBPP, TBA phân xưởng (tt)}
	TBA phân xưởng trọn bộ:
	\begin{itemize}
	\justifying
		\item Tổng công suất $S \leq 3.2MVA$ (bố trí hở).
		
		\item \alert{Khoảng cách giữa các máy biến áp:} \noibat{ngăn bằng rào}, cách nhau $\geq 10m$.
		\item Một buồng nên đặt một trạm có MBA dầu, $S\leq 6.5MVA$.
	\end{itemize}
\end{frame}

\begin{frame}{TBPP, TBA phân xưởng (tt)}
	\begin{itemize}
	\justifying
		\item Dưới MBA và TB có dầu: có hố thu dầu.
		\item ĐK dùng MC nhiều dầu trong ngăn: số MC $\leq 3$, khối lượng dầu của mỗi MC $\leq 60Kg$.
	\end{itemize}
\end{frame}

\begin{frame}{TBPP, TBA phân xưởng (tt)}
	Hệ thống thông gió:
	\begin{itemize}
	\justifying
		\item MBA làm việc trong \alert{môi trường bình thường:} \noibat{dùng không khí}.
		\item MBA làm việc trong \alert{môi trường nhiều bụi, chứa chất dẫn điện, ăn mòn:} lấy không khí bên ngoài hoặc phải lọc sạch.
	\end{itemize}
\end{frame}

\begin{frame}{TBPP, TBA phân xưởng (tt)}
	\begin{itemize}
	\justifying
		\item Sàn máy biến áp không thấp hơn sàn phân xưởng.
		\item Cửa ngăn MBA có dầu và MC nhiều dầu phải có giới hạn chịu lửa $0.6h$
		\item Bảo vệ TBA chống va chạm khi vận chuyển đồ trong phân xưởng.
	\end{itemize}
\end{frame}

\begin{frame}{TBPP, TBA phân xưởng (tt)}
	Qui định về lối đi trong TBA:
	\begin{itemize}
	\justifying
		\item Lối đi dọc theo tường: rộng $\geq 1m$.
		\item Lối đi ĐK: rộng $l_{\text{\textit{xe đẩy}}}$ $+0.6m$ (một dãy) hoặc $+0.8m$ (hai dãy) với đặt trong buồng.
		\item Chiều cao phòng: $h_{\text{\textit{max TBPP}}} + 0.8m$ đến trần nhà.
	\end{itemize}
\end{frame}

%--------------------------------------------------------------------------------
% Trang bị phân phối và TBA trên cột
\subsection*{Trang bị phân phối và TBA trên cột}
\begin{frame}{TBPP và TBA trên cột}
	\begin{itemize}
		\justifying
		\item MBA nối vào mạng điện qua \noibat{cầu chì và dao cách ly hoặc cầu chì tự rơi}.
		\item Độ cao đặt máy: $\geq 4m$ tính từ mặt đất đến phần dẫn điện.
		\item Dao cách ly, cầu chì tự rơi, phần tử cao áp còn mang điện: cao $\geq 2.5m$
	\end{itemize}
\end{frame}

\begin{frame}{TBPP và TBA trên cột (tt)}
	\begin{itemize}
		\justifying		
		\item Vị trí đóng ngắt dao cách ly phải thấy rõ ràng.
		\item Tủ điện hạ áp của MBA được đặt kín trong tủ.
		\item Bảo vệ dây dẫn điện tránh hư hỏng do tác động cơ khí.
	\end{itemize}
\end{frame}

\begin{frame}{TBPP và TBA trên cột (tt)}
	Bố trí TBA cách nhà có bậc chịu lửa:
	\begin{itemize}
		\justifying		
			\item Cách $\geq 3m$: với cấp $I-III$.
			\item Cách $\geq 5m$: với cấp $IV-V$.
			\item Nhưng nơi có nguy cơ xe cộ va vào: có biện pháp bảo vệ TBA.
	\end{itemize}
\end{frame}

%--------------------------------------------------------------------------------
% Bảo vệ chống sét
\subsection*{Bảo vệ chống sét}
\begin{frame}{Bảo vệ chống sét}
	\begin{itemize}
		\justifying
		\item Bảo vệ \alert{chống sét đánh trực tiếp}.
		\item Dùng \alert{kim thu sét} hoặc \alert{dây chống sét} bố trí trên kết cấu xây dựng.
		\item Nếu không lắp được kim thu sét thì dùng cột thu sét độc lập có $R_{\text{\textit{nđ}}} \leq 80\Omega$		
	\end{itemize}
\end{frame}

\begin{frame}{Bảo vệ chống sét (tt)}
	\begin{itemize}
		\item Bố trí CSV tại đầu ra các cuộn của MBA $6-35kV$.
		\item Có thể nối trực tiếp CSV với các cuộn mà không qua DCL.
		\item Sử dụng \alert{dây trần} nối CSV và MBA thì chiều dài $\leq 90m$.
	\end{itemize}
\end{frame}

%--------------------------------------------------------------------------------
% Hệ thống dầu
\subsection*{Hệ thống dầu}
\begin{frame}{Hệ thống dầu}
	\begin{itemize}
	\justifying
		\item \alert{Tổ chức hệ thống dầu tập trung:} thùng chứa dầu, xử lý dầu, trang bị lọc, tái sinh dầu,\ldots
		\item TBA \alert{có máy bù đồng bộ:} xây dựng \noibat{2 bể chứa dầu} không phụ thuộc vào số lượng và dung tích của bể dầu cách điện. $\tviet{V}{bể dầu} \geq 110\% \tviet{V}{bể dầu máy bù}$
	\end{itemize}
\end{frame}

\begin{frame}{Hệ thống dầu (tt)}
	\begin{itemize}
	\justifying
		\item \noibat{Không cần đặt} ống dầu cố định đến máy cắt dầu và MBA.
		\item Sử dụng \noibat{ống dẫn} và \noibat{thùng dầu di động} để xả nạp dầu.		
	\end{itemize}
\end{frame}

\begin{frame}{Hệ thống dầu (tt)}
	\begin{itemize}
	\justifying
		\item Ở nhà máy công nghiệp lớn, cần có \alert{hệ thống dầu riêng} (nếu lượng dầu lớn).
		\item \alert{Bể dầu} phải có: \noibat{bộ hô hấp không khí}, \noibat{bộ báo mức dầu}, \noibat{van xả}, \noibat{ống xả}.
	\end{itemize}
\end{frame}

%--------------------------------------------------------------------------------
% Lắp đặt máy biến áp lực
\subsection*{Lắp đặt máy biến áp lực}
\begin{frame}{Lắp đặt MBA lực}
	\begin{itemize}
	\justifying
		\item Chọn MBA theo khả năng: \alert{quá tải ngắn hạn và lâu dài}.
		\item Lắp đặt vị trí \alert{dễ quan sát mức dầu}, \alert{dễ tiếp cận relay hơi}.
		\item \alert{Lắp chống sét van $35kV$ trở xuống} trên nắp và thân MBA.
	\end{itemize}
\end{frame}

\begin{frame}{Lắp đặt MBA lực (tt)}
	\begin{itemize}
	\justifying
		\item \alert{MBA dầu} lắp đặt \noibat{đúng độ nghiêng yêu cầu} để khí đến được relay hơi.
		\item \alert{Lắp thùng dầu} trên \alert{kết cấu riêng} \noibat{không cản trở lấy MBA} ra khỏi móng. \alert{Relay hơi} bố trí gần MBA để tiếp cận thuận lợi và an toàn trên thang cố định.
	\end{itemize}
\end{frame}

\begin{frame}{Lắp đặt MBA lực (tt)}
	\begin{itemize}
	\justifying
		\item \alert{Ống phòng nổ} \noibat{không hướng gần} MBA.
		\item \alert{Khởi động thiết bị chữa cháy} bằng  \noibat{2 phương pháp} là \noibat{bằng tay} và \noibat{tự động}.
		\item Sàn của MBA dầu có độ nghiêng $2\%$ về hố thu dầu.
	\end{itemize}
\end{frame}

\begin{frame}{Lắp đặt MBA lực (tt)}
	\begin{itemize}
	\justifying
		\item MBA có \alert{hệ thống làm mát cưỡng bức} phải \noibat{tự khởi động được} và \noibat{tự dừng được hệ thống làm mát}.
		\item \alert{Đường ống dẫn dầu của bộ phận làm mát cưỡng bức} làm bằng \noibat{thép không gỉ}.
	\end{itemize}
\end{frame}

%--------------------------------------------------------------------------------
%--------------------------------------------------------------------------------
% Tài liệu tham khảo
\section*{Tài liệu tham khảo}
\begin{frame}{Tài liệu tham khảo}
\justifying
\hspace{1cm} \alert{Quy phạm trang bị điện, Phần III --Trang bị phân phối và Trạm biến áp}, Bộ Công Nghiệp -- NXB Lao Động -- Xã Hội, Năm 2006.
\end{frame}

%--------------------------------------------------------------------------------
%--------------------------------------------------------------------------------
% Lời cảm ơn
\section*{Lời cảm ơn}
\begin{frame}
\justifying
\Large \alert{Cảm ơn Thầy và các bạn đã theo dõi phần trình bày của nhóm!}
\end{frame}
\end{document}